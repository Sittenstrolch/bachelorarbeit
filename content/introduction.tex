\section{Introduction}

Nowadays, as economy has passed boundaries and not only people but also companies are connected throughout the world,
it has become impossible to keep track of everything. Companies interact with each other in lots of different ways like
being competitors, exchanging employees, using the same infrastructure and more. Some of these influences may create
similar struggles or needs for these businesses. Due to the growing presence of businesses online, especially on
social media platforms can be used to analyze companies behaviour.

\subsection{General Importance of Social Media}
Social Media Networks have increased in importance for companies. They use it for creating a closer relation to their customers,
for \emph{hiring} new employees, to take care of their contacts, to \emph{advertise} their offers or to \emph{look for new products}. Online business networks like
LinkedIn and Xing have a userbase of over 300 million \footnote{https://www.linkedin.com/about-us} and 9.2 million \footnote{https://corporate.xing.com/english/company/}
members.

So far new technologies, like the approach presented by Berger and Hennig\cite{n2o}, enable us to extract product relevant posts,
which express a demand, from social media networks for certain products. Using this information a sales representative can
actively engage with a new customer. This new form of selling products as a company also provides a lot more opportunities.
One of this opportunities will be developed within this thesis.

To be able to sell products using social media, potential new customers have to claim a need in a social network. As this
strategy is not widely spread yet, not all of the companies that have a demand do also claim it in a social network.

This thesis addresses this problem and presents an approach that is based on the main-thesis
that \emph{similar or strongly related businesses develop similar product needs}. The used way to prove this hypothesis is
to cluster companies and evaluate how far companies within one cluster develop the same needs.

By developing a tool for clustering and visualizing the formed groups it is possible to consider different clustering
algorithms and companies' characteristics for the purpose of getting the most accurate result for performing predictions.

After having grouped the businesses successfully this thesis develops a strategy to identify future claims to solve
the above problem.
