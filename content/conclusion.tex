\section{Future Work}
As this thesis is more a proof of concept than a final solution the main aspect of the
future work will be to improve the clustering by refering to more metrics as already mentioned
in section \ref{section:improvingResult}. Therefore it could be useful to also mention additional
datasources like the Compustat database which provides financial, statistical and market information
on companies throughout the world.

Since the datasets used for this thesis were entirely small in comparison to the information thats available
on LinkedIn or Crunchbase, the clustering tool should be able to handle millions of datasets. Useful predictions can
be generated from a lot of data only.

Indeed this thesis discussed and developed a model for identifying future product needs, but this approach still
needs to be implemented. So a tool for predicting future demands from companies will be necessary.

Current tools enable sales people to have an overview over their social networks and keep track of
companies that show interest in certain products. The tool developed with this thesis would provide
a way to multiply this demand. This would strongly increase the salesmen's efficiency.

Apart from predicting future demands this tool could also be used for marketing evaluations. As it provides
an overview of companies and visualizes the raised demands per cluster, evaluations like a spread rate or
number of companies reached can be generated. One could see how successful certain marketing campaigns were
and even better who exactly they may have reached. This would also provide information about a products user group.
All based on real demand tracking in social networks.

\clearpage

\section{Conclusion}
This thesis presented an approach to group companies, evaluate different clusters regarding
to their need development and predict demands of companies.

In addition to this thesis we developed a tool that allowed us to perform the tests for the evaluation.
We compared the two clustering algorithms bottom-up agglomorative clustering and the partitional kMeans clustering.
Both perfomed well and the results developed as expected. However, as we used only two features for the
partitional clustering the hierarchical algortithm produced the better result.

The data we used was sufficient enough to fullfill the needs for the used features. For further tests
it would be helpful to enrich the companies with even more data. Especially economic ones like the
income per year. This would allow to get even better clusters.

This thesis could not give the final answer to the problem, it is further more a
proof of concept that the primary objective to identify and predict future product
needs is possible. It reveals existing correlations between a company's demand development
and its different characteristics.

The opportunities are promising. Knowing companie's demands beforehand would create a big
advantage to the ones using it.
