\section{Related Work}

This chapter introduces two papers that also described an approach to create clusters of companies and
shortly explains their intention and strategy. Further more the key parts of each paper are going to be
highlighted and connected to the main-thesis.

\subsection{Statistical Approach for grouping companies}

Chen, Gnanadesikan and Kettenring \cite{StatisticalGrouping} already described in 1974 an approach to group companies in
their paper ``Statistical methods for grouping corporations''. Their general objecitve was to ``detect, describe and
distinguish relatively homogeneous groups of companies''

In their paper they compared a classification of companies by the use of a knowledgebase to a computed cluster analysis.
As proximity measures they used fourteen self chosen normalized economic statistics like dividends per share, number of employees in proportion
to net plant or the correlation of net sales to net plant, to mention only some of them.

They analyzed companies from 5 different industries and were able to assign most of the companies to the right cluster, by only
considering their economic measurements. As a consequence companies that belong to the same industry mostly act similar
regarding to their economic statistics. This conclusion confirms the main-thesis insofar as businesses of the same industry
may act in a similar way.

\subsection{Economic Cluster Analysis}

In their paper ``Homogenous groups and the testing of economic hypothesis'' Elton and Gruber \cite{EconomicClusterAnalysis}
explore cluster analysis for the disaggregation of economic data into meaningful groups. Their main objective was to
show the importance of grouping companies and describe ways in order to test financial hypotheses.
One key aspect was to get better results by decomposing measurements to avoid certain characteristics that may
be represented by multiple variables.

After explaining how to decompose variables into a new set of varibles without any interferences by the means of a
principal components analysis they discussed criterias for grouping like group compactness.

The key aspect for the main thesis is the prevention of possible interferences that can exist between some grouping criteria.
Because analyzing financial values can give us information about a firm's possible buying behaiviour its important to choose the
criterias correctly in order to weight the values correctly.
