\section{Background}

To identify companies with a similar demand, it is crucial to understand how companies develop product needs. This chapter will shortly describe Porters Theory \cite{CompanyClusters} of
economic clusters and some of his conclusions. Furthermore it is going to explain a subpart of
Webster and Wind's model \cite{BusinessBuyingBehavior} of organizational buying behaviour. They describe environmental influences to which
companies are exposed to.

Another important work that is necessary to prove the main thesis, that strongly related businesses develop similar product needs,
is the lead extraction from social networks.
This approach helps to create a dataset of raised company demands over a time-period. Having this information makes
it possible to detect raised needs within a cluster over time.


\subsection{Economic Clusters}
An economic cluster is a group of companies that are strongly related to each other. This relations could exist
through the same industry, a similar company size, the same products or other indicators.

According to Michael E. Porter \cite{CompanyClusters} "Clusters are geographic concentrations of
interconnected companies and institutions in a particular field"

These clusters comprise different companies of an industry, including suppliers of specialized inputs such as components,
machinery and services, and providers of specialized infrastructure. A cluster contains linkages and complemetaries
that are most important to competition.

A vital part of a cluster is an existing competitive attitude. It can survive only if belonging companies try to
exceed each other. The quality with which companies compete in a perticular location is influenced by the quality
of the local business environment. High quality goods can not be produced without good suppliers or an established
transportation infrastructure.

This leads to the other important part of a cluster which is the cooperation. Companies can learn from each other
and build on an existing infrastructure  of suppliers and providers for goods and services which belong to the
cluster as well.

Porter emphasizes the importance of a companie's location for its success, even in times of global markets and faster
transportation.

Companies within a cluster are closely related. They depend on each other and are highly influenced by the cluster.
As the cluster changes, companies change too. If companies are influenced by the cluster, which is nothing else
than companies that are related through their industry and location, than they will also develop together regarding
their product needs. This supports our initial assumption that strongly related companies develop similar demands.

\subsection{Organizational Buying Behavior}
Webster and Wind~\cite{BusinessBuyingBehavior} described a general model to explain organizational buying behavior.

The model addresses the influence factors that may raise new needs as well as the
decision process within the company and the actual transaction. The influence factors are mostly relevant
here. Following 6 types of environmental influences are mentioned by them:

\begin{itemize}
  \item Economic (unemployment,economic growth)
  \item Political (public subsidies)
  \item Physical (goegraphic, climate, ecological)
  \item Technological (internet infrastructure)
  \item Legal (law restrictions)
  \item Cultural (Diverse working attitudes)
\end{itemize}

These influences are exerted through several institutions like suppliers, customers, competitors,
governments, trade unions and political parties. They have their impact in four different ways.

First of all they define the availability of goods and services. Especially physical, technological and economic
influences affect this impact. For example solar power plants are better situated in areas that provide a lot of sunlight
like a desert.

Second they define general business conditions as the rate of economic growth, the level of national income,
interest rates, and umemployment. Economic and political forces are the most dominant influences here. Businesses that
need many employees are better situated in regions with higher unemployment and educated people.

Third, environmental factors define values and norms of interorganizational and interpersonal relationships between
most of the market's participants like buyers, sellers, competitors and governments. Values and norms may be
specified by law. But most important are cultural, social, legal and political forces.

Finally, information flow into buying organizations are influenced by environmental forces too. Most vitally
to mention here is the \"flow of marketing communications from potential suppliers, through the mass media and
through other personal and impersonal channel \". A variety of physical, technological, economic, and cultural
factors are showing their effect here.

These influences are important to find measurements that group companies with similar circumstances. Ignoring
them would lead to false results that do not represent companies that are exposed to the same influences. Only companies
dealing with the same challenges would develop similar demands.

The challenge concerning the different influences is to find good measurements for each of them. Cultural, legal, physical,
and political influences are especially tough to find. One attempt to cover those is to use a company's local information.
A place can be defined through the country and therefore unites the political influence by the country and city, as well as
the geographic conditions and the cultural attitudes of the people living there. The other two left influences can
be described more easily by several publicly available indicators like the gross domestic product or the Human Development
Index. The data used in this Thesis will mainly cover a company's location and its own economic values.

\subsection{Generating Leads from social networks}
Berger and Hennig's approach of converting social media posts to leads~\cite{n2o} helps to get a measurement of
raised needs in companies.

They extract posts from social media, classify them with a two-stage classifier that sorts the posts by demand and
tags certain products based on an already established knowledgebase created for the products.

Having the information of needs in companies makes it possible to address only companies that want to buy certain
products.

Their two-stage classification not only makes it possible to analyse a general demand-evolvement for companies,
but furthermore special products, which allows the evaluation of the thesis to be even more meaningful.

\subsection{Clustering Algorithms}
To accomplish the task of finding relationships between two or more companies, for example by grouping them, several
algorithms are known. This part shortly describes and compares some of the major strategies to find the most convenient
in order to cluster companies.

\subsubsection{Clustering characteristics}
Existing algorithms can be characterized by the following properties: \cite{jain+dubes}
\begin{itemize}
  \item \emph{Exclusive or nonexclusive}. An exclusive
 classification applies an entity to exactly one cluster, whereas a nonexclusive approach can assign multiple clusters
 for one entity.
  \item \emph{Intrinsic and extrinsic clustering}. Intrinsic clustering only
uses the calculated proximity matrix for asigning clusters. An extrinsic strategy would additionaly use previously
taged values that may already provide some kind of clustering. This strategy is used to find different characteristics
that are distinct for the different taged groups.
  \item \emph{Hierarchical and paritional}. Only exclusive and intrinsic algorithms are subdivided in this two categories.
  A hierarchical algorithm is a sequence of partitions. It produces multiple clusterings, one per sequence, going from
  one cluster (contains all entities) to as many clusters as entities exist (one cluster per entity), which is the top-down
  approach called divisive. The bottom-up version works the opposite direction and is called agglomerative. The number of
  clusters does not have to be known for the algorithm but in return one has to select the most appropriate division
  produced by this algorithm.
  As against a partitional attempt consists of only one single partition. An partitional approach needs to know the number
  of clusters at the beginning. Then it chooses, more or less randomly, the cluster centres and applies the other entities.
  Thus a hierarchical classification is a special sequence of partitional classifications.

\end{itemize}

In lots of cases clustering algorithms are combined to get better results. The combination may allow to recognize outliers
and reduce their impact on defining wrong clusters, or to determine a better approximation to the number of clusters.
An example could be to first perform a hierarchical clustering to determine a good count of clusters, and afterwards to perform
a partitional clustering in order to get improve the result.

\subsubsection{Used Clustering Algorithm}
\label{clusteringDiscussion}

Some clustering approaches need to know the number of clusters. Of course one could estimate a number of clusters
by considering the number of industries as well as the number of different locations for each industry of a company, but this would
still be an approximation to the number, which by the way would get invalid by adding more companies.
Hierarchical algorithms have the advantage that they do not need to know the number of used clusters beforehand.
This leads to the problem to figure out which of the multiple generated
partitions should be used. So it is necessary to have a measurement for partitions to find out which one
works best.

Furthermore the used clustering algorithms has to be \emph{exclusive} and \emph{intrinsic}. It would not be on purpose to find
characteristics on predefined groups but rather to define groups of companies. An exclusive approach would provide
the information to which cluster a company belongs and that is what we are looking for.


The aim to explore and furthermore predict the need evolvement could be achieved by grouping strongly connected companies.
Companies that belong to one cluster should ideally have the same demands. To match the main thesis its important to find
correlations between closeness of companies and their needs. Especially its important that a cluster evolves exactly one same
need, according to the assumption we make that each company only raises one need. This requirement makes it possible to allow predictions on a cluster's demand evolvement.

Therefore the approach will be to group companies in that way that each cluster has on major product.
In this thesis a bottom-up agglomerative hierarchical algorithm will be used as
well as the partitional kMeans algorithm in comparison. Both algorithms are \emph{exclusive} and \emph{intrinsic}.

As the hierarchical algorithm produces different possible clusters a way to determine the best clusters is necessary. One clustercombination
has to fullfill the following characteristics:
\begin{itemize}
  \item All the clusters have a strong increase of exactly one demand each
  \item A cluster contains only companies that do not have the maximum possible proximity \footnote{For detailed information on the proximity calculation see section \ref{companyProximity}}
\end{itemize}

For the number of clusters to pass to the partitional clustering we use different values to test what works best.
