\section{Background}

It is crucial to understand how companies develop product needs. This chapter will shortly describe Porters Theory of
economic clusters and some of his conclusions. Furthermore it is going to explain a subpart of
Webster and Wind's model of organizational buying behaviour. They describe environmental influences to which
companies are exposed to.

Another important work that is necessary to prove the main thesis, that strongly related businesses develop similar product needs,
is the lead extraction from social networks.
This approach helps to create a dataset of raised company demands over a time-period. Having this information makes
it possible to detect raised needs within a cluster over time.


\subsection{Economic Clusters}

According to Michael E. Porter \cite{CompanyClusters} "Clusters are geographic concentrations of
interconnected companies and institutions in a particular field"

These clusters include different companies of an industry, including suppliers of specialized inputs such as components,
machinery and services, and providers of specialized infrastructure. A cluster contains linkages and complemetaries
that are most important to competition.

A vital part of a cluster is an existing competitive attitude. It can survive only if belonging companies try to
exceed each other. The quality with which companies compete in a perticular location is influenced by the quality
of the local business environment. High quality goods can not be produced without good suppliers or an established
transportation infrastructure.

This leads to the other important part of a cluster which is the cooperation. Companies can learn from each other
and build on an existing infrastructure  of suppliers and providers for goods and services which belong to the
cluster as well.

Porter emphasizes the importance of a companie's location for its success, even in times of global markets and faster
transportation.

Companies within a cluster are closely related. They depend on each other and are highly influenced by the cluster.
As the cluster changes, companies change too. If companies are influenced by the cluster, which is nothing else
than companies that are related through their industry and location, than they will also develop together regarding
their product needs.

\subsection{Organizational Buying Behavior}
Webster and Wind~\cite{BusinessBuyingBehavior} described a general model to explain organizational buying behavior.

The model addresses the influence factors that may raise new needs as well as the
decision process within the company and the actual transaction. The influence factors are mostly relevant
here. Following 6 types of environmental influences are mentioned by them:

\begin{itemize}
  \item Economic (unemployment,economic growth)
  \item Political (public subsidies)
  \item Physical (goegraphic, climate, ecological)
  \item Technological (internet infrastructure)
  \item Legal (law restrictions)
  \item Cultural (Diverse working attitudes)
\end{itemize}

These influences are exerted through several institutions like suppliers, customers, competitors,
governments, trade unions and political parties. They have their impact in four different ways.

First of all they define the availability of goods and services. Especially physical, technological and economic
influences affect this impact.

Second they define general business conditions as the rate of economic growth, the level of national income,
interest rates, and umemployment. Economic and political forces are the most dominant influences here.

Third, environmental factors define values and norms of interorganizational and interpersonal relationships between
most of the market's participants like buyers, sellers, competitors and governments. Values and norms may be
specified by law. But most important are cultural, social, legal and political forces.

Finally, information flow into buying organizations are influenced by environmental forces too. Most vitally
to mention here is the \"flow of marketing communications from potential suppliers, through the mass media and
through other personal and impersonal channel \". A variety of physical, technological, economic, and cultural
factors are showing their effect here.

These influences are important to find measurements that group companies with similar circumstances. Ignoring
them would lead to false results that do not represent companies that are exposed to the same influences. Only companies
dealing with the same challenges would develop similar demands.

\subsection{Generating Leads from social networks}
Berger and Hennig's approach of converting social media posts to leads~\cite{n2o} helps to get a measurement of
raised needs in companies.

They extract posts from social media, classify them with a two-stage classifier that sorts the posts by demand and
tags certain products based on an already established knowledgebase created for the products.

Having the information of needs in companies makes it possible to address only companies that want to buy certain
products.

Their two-stage classification not only makes it possible to analyse a general demand-evolvement for companies,
but furthermore special products, which allows the evaluation of the thesis to be even more meaningful.

\subsection{Clustering Algorithms}
To accomplish the task of finding relationships between two or more companies, for example by grouping them, several
algorithms are known. This part shortly describes and compares some of the most known ones to find the most convenient
in order to proof the main thesis.

Different Algorithms may belong to some of the following categories: \cite{jain+dubes}
\begin{itemize}
  \item \emph{Exclusive or nonexclusive}. An exclusive
 classification applies an entity to exactly one cluster, whereas a nonexclusive approach can assign multiple clusters
 for one entity.
  \item \emph{Intrinsic and extrinsic clustering}. Intrinsic clustering only
uses the calculated proximity matrix for asigning clusters. An extrinsic strategy would additionaly use previously
taged values that may already provide some kind of clustering. This strategy is used to find different characteristics
that are distinct for the different taged groups.
  \item \emph{Hierarchical and paritional}. Only exclusive and intrinsic algorithms are subdivided in this two categories.
  A hierarchical algorithm is a sequence of partitions. It produces multiple clusterings, one per sequence, going from
  one cluster (contains all entities) to as many clusters as entities exist (one cluster per entity), which is the top-down
  approach called divisive. The bottom-up version works the opposite direction and is called agglomerative. The number of
  clusters does not have to be known for the algorithm but in return one has to select the most appropriate division
  produced by this algorithm.
  As against a partitional attempt consists of only one single partition. An partitional approach needs to know the number
  of clusters at the beginning. Then it chooses, more or less randomly, the cluster centres and applies the other entities.
  Thus a hierarchical classification is a special sequence of partitional classifications.

\end{itemize}

In lots of cases clustering algorithms are combined to get better results. The combination may allow to recognize outliers
and reduce their impact on defining wrong clusters, or to determine a better approximation to the number of clusters.
