\documentclass[%
a4paper,
DIV12,
2.5headlines,
bigheadings,
titlepage,
openbib,
%draft
]{scrartcl}

%%% PACKAGES
\usepackage[ngerman, english]{babel}
%% FONTS


\usepackage[T1]{fontenc}
\usepackage{geometry}
\usepackage[utf8]{inputenc}
\usepackage{mathpazo}
\usepackage{helvet}
\usepackage{courier}
\usepackage{eurosym}
\usepackage{amsmath}
\usepackage{courier}
\usepackage{scrpage2}
\usepackage{graphicx}
\usepackage{xcolor}
\usepackage{multirow}
\usepackage{varioref}
\usepackage{babelbib}
\usepackage{makeidx}
\usepackage{tabularx}
\usepackage{floatflt}
\usepackage[pdftex, colorlinks, linktocpage, linkcolor=black, citecolor=black, urlcolor=black]{hyperref}
\usepackage{array}

\newcommand{\head}[2]{\multicolumn{1}{>{\centering\arraybackslash}p{#1}}{\textbf{#2}}}

\graphicspath{ {images/} }

\pagestyle{scrheadings}


\geometry{a4paper, top=55mm, left=40mm, right=35mm, bottom=40mm,
headsep=10mm, footskip=22mm}
\linespread {1.25}
%%% COMMANDS

	%%%%%%%%%%%%%%%%%%
	% Autor eintragen
	\newcommand{\theauthor}{Markus Petrykowski}
	%%%%%%%%%%%%%%%%%%
	% Matrikelnummer eintragen
	\newcommand{\matrnr}{761918}
	%%%%%%%%%%%%%%%%%%
	% Titel eintragen
	\newcommand{\thetitle}{Identifying Future product needs by Clustering Companies}

%%% COLORS√
\input{utils/hpicolors}

%%% OTHER INPUTS
\input{utils/commands}
\input{utils/environments}
\newcommand{\frontmatter}{\pagenumbering{roman}}
\newcommand{\mainmatter}{\pagenumbering{arabic}\setcounter{page}{1}}
%%% INCLUDE ONLY
\setlength{\parindent}{0cm}
\setlength{\parskip}{0.25cm}
%%% DOCUMENT
\begin{document}
	%%% HEADER AND FOOTTITLES
	%\selectlanguage{ngerman}
	\selectlanguage{english} % {ngerman}
	\automark{section}
	\ohead{\includegraphics[height=1.3cm,clip,viewport={0 60 250 180}]{utils/hpi_logo.pdf}}
	\chead{}
	\ihead{\headmark}
	\setheadsepline{1.0pt}[\color{hpigrey}]
	%%% TITLEPAGE
	\hypersetup{%
		pdftitle	= {\thetitle},
		pdfsubject	= {Bachelor's Thesis},
		pdfauthor	= {\theauthor},
		pdfcreator	= {PDFLaTeX},
		pdfproducer	= {LaTeX with hyperref and thumbpdf}
			   }

		\titlehead{
	%\parbox[b]{10cm}{\sffamily{\Large Hasso Plattner Institut}  \\Prof.~Dr.~Helmertstra�e~2-3 \\14482 Potsdam}
	\centering
	\includegraphics[height=4cm]{utils/hpi_logo_text.pdf}

	}		\subject{{\LARGE Bachelor's Thesis}\\}
	\title{\thetitle}
	\subtitle{\thesubtitle}
	\author{{\small by}\\\textbf{\theauthor}}
	%\dedication{Widmung\\mit mehreren\\Zeilen.}
	\date{Potsdam, Juni 2015}
	\publishers{
		\textbf{Supervisor}\\
		\vskip1em
		Prof. Dr. Christoph Meinel\\

		\vskip2em
		\textbf{Internet-Technologies and Systems Group}
		}
	\frontmatter
	\maketitle

	%\input{titlepage_german}




	\section*{Disclaimer}

I certify that the material contained in this dissertation is my own work and does not contain significant portions of unreferenced or unacknowledged material. I also warrant that the above statement applies to the implementation of the project and all associated documentation.\\\\
Hiermit versichere ich, dass diese Arbeit selbst\"{a}ndig verfasst wurde und dass keine anderen Quellen und Hilfsmittel als die angegebenen benutzt wurden. Diese Aussage trifft auch f\"{u}r alle Implementierungen und Dokumentationen im Rahmen dieses Projektes zu.

	\begin{flushleft}
	Potsdam, \today
	\end{flushleft}
	\begin{picture}(150,70)
		\put(0,15){\line(1,0){150}}
		\put(0,0){(\theauthor)}
	\end{picture}
	\clearpage

	%%% Abstract
	\myabstract{%
	% deutsche Zusammenfassung
	Da Unternehmen gerade in sozialen Netzwerken immer aktiver werden, ist es besonders interessant geworden diese
	Auftritte zu analysieren und Bedarfsanalysen der Unternehmen damit vorzunehmen.
	Diese Bachelorarbeit beschreibt zu Beginn eine Herangehensweise zur Gruppierung von Unternehmen. Die dabei entstandenen
	so genannten Cluster bestehen aus eng in Verbindung stehenden Unternehmen. Diese entstandenen Gruppen werden anschließend
	in Hinblick auf ihre Bedarfsentwicklung evaluiert. Daraufhin kann mit Hilfe der ausgesuchten Cluster der Bedarf von
	Unternehmen verhergesehen werden.
	Die Auswertung des Clusterings zeigt signifikante Korrelationen zwischen der Bedarfsentwicklung von Unternehmen und ihrer
	Merkmale auf.

	}{

	%
	% englischer abstract
	As companies are becoming more and more active in social networks, it has become interesting to evaluate these
	social media interactions and extracting demands from them.
	This thesis describes an approach that first groups companies. So called clusters consist of interconnected businesses.
	Second it evaluates the formed clusters regarding to their need development. Third it takes the chosen cluster and
	performs demand predictions of companies. The evaluation of the grouping shows significant correlations for the demand
	development of companies.
	}

	%%% TOC
	\tableofcontents
	\clearpage
	%%% INCLUDES
	\mainmatter


	%%% Introduction
	% Context of this work
	% Central problem
	% Motivation, why is this topic of importance
	% short Overview of each section
	\section{Introduction}

\subsection{Outline}

Nowadays, as economy has passed boundaries and not only people but also companies are connected throughout the world,
it has become impossible to keep track of everything. Companies interact with each other in lots of different ways like
being competitors, exchanging employees, using the same infrastructure and more. Some of these influences may create
similar struggles or needs for these businesses. Due to the growing presence of businesses online, especially
social media platforms can be used to analyze companies behaviour.

So far webtechnologies enable us to extract product relevant posts, which express a demand,
from social media networks for certain products. Using this information a sales representative can
actively engage with a new customer. This thesis is going to present an approach that is based on the assumption
that similar or strongly related businesses develop similar product needs. We are going to prove this assumption,
explore existing correlations and develop a strategy to identify future claims.

	\clearpage

	\section{Background}

To identify companies with a similar demand, it is crucial to understand how companies develop product needs. This chapter will shortly describe Porters Theory \cite{CompanyClusters} of
economic clusters and some of his conclusions. Furthermore it is going to explain a subpart of
Webster and Wind's model \cite{BusinessBuyingBehavior} of organizational buying behaviour. They describe environmental influences to which
companies are exposed to.

Another important work that is necessary to prove the main thesis, that strongly related businesses develop similar product needs,
is the lead extraction from social networks.
This approach helps to create a dataset of raised company demands over a time-period. Having this information makes
it possible to detect raised needs within a cluster over time.


\subsection{Economic Clusters}
An economic cluster is a group of companies that are strongly related to each other. This relations could exist
through the same industry, a similar company size, the same products or other indicators.

According to Michael E. Porter \cite{CompanyClusters} "Clusters are geographic concentrations of
interconnected companies and institutions in a particular field"

These clusters comprise different companies of an industry, including suppliers of specialized inputs such as components,
machinery and services, and providers of specialized infrastructure. A cluster contains linkages and complemetaries
that are most important to competition.

A vital part of a cluster is an existing competitive attitude. It can survive only if belonging companies try to
exceed each other. The quality with which companies compete in a perticular location is influenced by the quality
of the local business environment. High quality goods can not be produced without good suppliers or an established
transportation infrastructure.

This leads to the other important part of a cluster which is the cooperation. Companies can learn from each other
and build on an existing infrastructure  of suppliers and providers for goods and services which belong to the
cluster as well.

Porter emphasizes the importance of a companie's location for its success, even in times of global markets and faster
transportation.

Companies within a cluster are closely related. They depend on each other and are highly influenced by the cluster.
As the cluster changes, companies change too. If companies are influenced by the cluster, which is nothing else
than companies that are related through their industry and location, than they will also develop together regarding
their product needs. This supports our initial assumption that strongly related companies develop similar demands.

\subsection{Organizational Buying Behavior}
Webster and Wind~\cite{BusinessBuyingBehavior} described a general model to explain organizational buying behavior.

The model addresses the influence factors that may raise new needs as well as the
decision process within the company and the actual transaction. The influence factors are mostly relevant
here. Following 6 types of environmental influences are mentioned by them:

\begin{itemize}
  \item Economic (unemployment,economic growth)
  \item Political (public subsidies)
  \item Physical (goegraphic, climate, ecological)
  \item Technological (internet infrastructure)
  \item Legal (law restrictions)
  \item Cultural (Diverse working attitudes)
\end{itemize}

These influences are exerted through several institutions like suppliers, customers, competitors,
governments, trade unions and political parties. They have their impact in four different ways.

First of all they define the availability of goods and services. Especially physical, technological and economic
influences affect this impact. For example solar power plants are better situated in areas that provide a lot of sunlight
like a desert.

Second they define general business conditions as the rate of economic growth, the level of national income,
interest rates, and umemployment. Economic and political forces are the most dominant influences here. Businesses that
need many employees are better situated in regions with higher unemployment and educated people.

Third, environmental factors define values and norms of interorganizational and interpersonal relationships between
most of the market's participants like buyers, sellers, competitors and governments. Values and norms may be
specified by law. But most important are cultural, social, legal and political forces.

Finally, information flow into buying organizations are influenced by environmental forces too. Most vitally
to mention here is the \"flow of marketing communications from potential suppliers, through the mass media and
through other personal and impersonal channel \". A variety of physical, technological, economic, and cultural
factors are showing their effect here.

These influences are important to find measurements that group companies with similar circumstances. Ignoring
them would lead to false results that do not represent companies that are exposed to the same influences. Only companies
dealing with the same challenges would develop similar demands.

The challenge concerning the different influences is to find good measurements for each of them. Cultural, legal, physical,
and political influences are especially tough to find. One attempt to cover those is to use a company's local information.
A place can be defined through the country and therefore unites the political influence by the country and city, as well as
the geographic conditions and the cultural attitudes of the people living there. The other two left influences can
be described more easily by several publicly available indicators like the gross domestic product or the Human Development
Index. The data used in this Thesis will mainly cover a company's location and its own economic values.

\subsection{Generating Leads from social networks}
Berger and Hennig's approach of converting social media posts to leads~\cite{n2o} helps to get a measurement of
raised needs in companies.

They extract posts from social media, classify them with a two-stage classifier that sorts the posts by demand and
tags certain products based on an already established knowledgebase created for the products.

Having the information of needs in companies makes it possible to address only companies that want to buy certain
products.

Their two-stage classification not only makes it possible to analyse a general demand-evolvement for companies,
but furthermore special products, which allows the evaluation of the thesis to be even more meaningful.

\subsection{Clustering Algorithms}
To accomplish the task of finding relationships between two or more companies, for example by grouping them, several
algorithms are known. This part shortly describes and compares some of the major strategies to find the most convenient
in order to cluster companies.

\subsubsection{Clustering characteristics}
Existing algorithms can be characterized by the following properties: \cite{jain+dubes}
\begin{itemize}
  \item \emph{Exclusive or nonexclusive}. An exclusive
 classification applies an entity to exactly one cluster, whereas a nonexclusive approach can assign multiple clusters
 for one entity.
  \item \emph{Intrinsic and extrinsic clustering}. Intrinsic clustering only
uses the calculated proximity matrix for asigning clusters. An extrinsic strategy would additionaly use previously
taged values that may already provide some kind of clustering. This strategy is used to find different characteristics
that are distinct for the different taged groups.
  \item \emph{Hierarchical and paritional}. Only exclusive and intrinsic algorithms are subdivided in this two categories.
  A hierarchical algorithm is a sequence of partitions. It produces multiple clusterings, one per sequence, going from
  one cluster (contains all entities) to as many clusters as entities exist (one cluster per entity), which is the top-down
  approach called divisive. The bottom-up version works the opposite direction and is called agglomerative. The number of
  clusters does not have to be known for the algorithm but in return one has to select the most appropriate division
  produced by this algorithm.
  As against a partitional attempt consists of only one single partition. An partitional approach needs to know the number
  of clusters at the beginning. Then it chooses, more or less randomly, the cluster centres and applies the other entities.
  Thus a hierarchical classification is a special sequence of partitional classifications.

\end{itemize}

In lots of cases clustering algorithms are combined to get better results. The combination may allow to recognize outliers
and reduce their impact on defining wrong clusters, or to determine a better approximation to the number of clusters.
An example could be to first perform a hierarchical clustering to determine a good count of clusters, and afterwards to perform
a partitional clustering in order to get improve the result.

\subsubsection{Used Clustering Algorithm}
\label{clusteringDiscussion}

Some clustering approaches need to know the number of clusters. Of course one could estimate a number of clusters
by considering the number of industries as well as the number of different locations for each industry of a company, but this would
still be an approximation to the number, which by the way would get invalid by adding more companies.
Hierarchical algorithms have the advantage that they do not need to know the number of used clusters beforehand.
This leads to the problem to figure out which of the multiple generated
partitions should be used. So it is necessary to have a measurement for partitions to find out which one
works best.

Furthermore the used clustering algorithms has to be \emph{exclusive} and \emph{intrinsic}. It would not be on purpose to find
characteristics on predefined groups but rather to define groups of companies. An exclusive approach would provide
the information to which cluster a company belongs and that is what we are looking for.


The aim to explore and furthermore predict the need evolvement could be achieved by grouping strongly connected companies.
Companies that belong to one cluster should ideally have the same demands. To match the main thesis its important to find
correlations between closeness of companies and their needs. Especially its important that a cluster evolves exactly one same
need, according to the assumption we make that each company only raises one need. This requirement makes it possible to allow predictions on a cluster's demand evolvement.

Therefore the approach will be to group companies in that way that each cluster has on major product.
In this thesis a bottom-up agglomerative hierarchical algorithm will be used as
well as the partitional kMeans algorithm in comparison. Both algorithms are \emph{exclusive} and \emph{intrinsic}.

As the hierarchical algorithm produces different possible clusters a way to determine the best clusters is necessary. One clustercombination
has to fullfill the following characteristics:
\begin{itemize}
  \item All the clusters have a strong increase of exactly one demand each
  \item A cluster contains only companies that do not have the maximum possible proximity \footnote{For detailed information on the proximity calculation see section \ref{companyProximity}}
\end{itemize}

For the number of clusters to pass to the partitional clustering we use different values to test what works best.

	\clearpage

	\section{Related Work}

This chapter introduces two papers that also described an approach to create clusters of companies and
shortly explains their intention and strategy. Further more the key parts of each paper are going to be
highlighted and connected to the main-thesis.

\subsection{Statistical Approach for grouping companies}

Chen, Gnanadesikan and Kettenring \cite{StatisticalGrouping} already described in 1974 an approach to group companies in
their paper ``Statistical methods for grouping corporations''. Their general objecitve was to ``detect, describe and
distinguish relatively homogeneous groups of companies''

In their paper they compared a classification of companies by the use of a knowledgebase to a computed cluster analysis.
As proximity measures they used fourteen self chosen normalized economic statistics like dividends per share, number of employees in proportion
to net plant or the correlation of net sales to net plant, to mention only some of them.

They analyzed companies from 5 different industries and were able to assign most of the companies to the right cluster, by only
considering their economic measurements. As a consequence companies that belong to the same industry mostly act similar
regarding to their economic statistics. This conclusion confirms the main-thesis insofar as businesses of the same industry
may act in a similar way.

\subsection{Economic Cluster Analysis}

In their paper ``Homogenous groups and the testing of economic hypothesis'' Elton and Gruber \cite{EconomicClusterAnalysis}
explore cluster analysis for the disaggregation of economic data into meaningful groups. Their main objective was to
show the importance of grouping companies and describe ways in order to test financial hypotheses.
One key aspect was to get better results by decomposing measurements to avoid certain characteristics that may
be represented by multiple variables.

After explaining how to decompose variables into a new set of varibles without any interferences by the means of a
principal components analysis they discussed criterias for grouping like group compactness.

The key aspect for the main thesis is the prevention of possible interferences that can exist between some grouping criteria.
Because analyzing financial values can give us information about a firm's possible buying behaiviour its important to choose the
criterias correctly in order to weight the values correctly.

	\clearpage

	\section{Company Clustering Algorithm}

\subsection{Data}
To determine clusters of companies, its necessary to have a data-set that contains the relevant information
for a company, and has to be big enough to get meaningful results.

\begin{figure}[ht]
\includegraphics[scale=0.5]{sources_dataschema.png}
\centering
\caption{Comparison of dataschemas}
\label{fig:sourcesSchemas}
\end{figure}

\subsubsection{Datasources}
To ensure a good quality the data-sets were extracted from two different sources, LinkedIn and Crunchbase.

LinkedIn is a social business network with over 300 million user,\footnote{https://www.linkedin.com/about-us, 28th of June 2015} with people from all over the world. Apart
from user-profiles it also contains company-profiles with properties like year of foundation, industry or
number of employees. The information are maintained by the companies itself.

Crunchbase is an open database containing startup-activity and company information.\footnote{https://info.crunchbase.com/about/ 28th of June 2015} Company-datasets contain information
like employees, competitors, industry and basic information as well. Like the wikipedia information can be maintained by everyone,
which could lead to frequently updated information on the one hand, and to wrong information on the other hand.

Figure \ref{fig:sourcesSchemas} shows a subset of attributes of companies that are provided by each source. \footnote{More detailed information can be found on http://data.crunchbase.com/v3/docs/organization and https://developer.linkedin.com/docs/fields/company-profile }
The characteristics that represent information to conclude a companies demand are printed bold. \footnote{Regarding to influencing factors
and a companies environment in chapter 2 and 3. See also Chapter 4.3} Both datasets provide similar information but with a different structure.
For example the number of employees. Crunchbase provides 2 attributes one for the mininum value and one for the maximum value as integers
whereas LinkedIn delivers a string like ``1001-5000'' which requires further processing to extract the same information.

\begin{figure}[ht]
\includegraphics[scale=0.3]{companyCountries.png}
\centering
\caption{Distribution of companies per country (Total 236,235 companies)}
\label{fig:countryDistribution}
\end{figure}


\subsection{Dataprocessing}
Because both sources have different advantages and information and as mentioned in the last section a different structure,
it makes sense to combine both datasets into one, that covers all the necessary information needed for clustering,
and has one defined dataschema.

The biggest problem in combining these two datasets is finding the right corresponding company in the respectively
other dataset.
The used approach was to join to datasets on a 100\% match of both companynames. If companies have slightly different names
in both sets, they will be matched if they have the same website url given. Otherwise a new entry will be created in
the resulting dataset. This resulted in a dataset of 236,235 companies. As you can see in Figure \ref{fig:countryDistribution}
the most companies are located in the United States. The dataset contains companies from 220 countries.

[ Maybe add industries as well. ]




% Challenge: Finding a combination of Features which matches the closeness of companies
% to the need's development
\subsection{Company Features}
\label{companyFeatures}
Features are variables or a combination of variables that can describe certain characteristics of an entity. Using the right
features is essential to prove that strongly related businesses develop similar product needs. In this case the features have
to describe characteristics that influence a companies buying behaviour.

Regarding to Porter \cite{CompanyClusters} a company's \emph{location} has a high influence on how it acts. Companies will often rather know what happens next to
them than at a totally different place. Steps taken by companies right next to each other will have a higher impact on
how each of them reacts to particular circumstances, especially purchases made by one of the companies may lead to an economic
advantage. Other companies are then forced to close this gap by doing similar purchases.

Of course the location is important but has less impact if the companies next to each other do not compete somehow.
Referring to Porter \cite{CompanyClusters} companies of the same \emph{industry} are often shaped in clusters at one location.
They are using the same infrastructure and increasing the clusters know how.

So the first two features that cause the highest influence from one company to another are a company's location and its industry.

An increasing number of employees within a company leads to a higher complexity. Also bigger companies have other needs and higher expenses than
smaller ones have. Therefore companies of similar size are more related to each other than to smaller sized companies.
This leads us to the third feature, a company's size measured by its \emph{number of employees}.

According to Webster and Wind \cite{BusinessBuyingBehavior} companies are exposed to 6 different influences. These influences are already covered
by the selected features. For example by selecting a company's location the legal, economic and political influences which
are the strongest ones are considered.

Other characteristics mentioned in chapter 4.1 could also be used as features. But as the selected 3 features cover all the aspects
discussed during the economic background, there is no need for more features at the moment. A comparison of results using different combinations of
features could be part of future work. This thesis focuses on finding a correlation between the closeness of companies and their
demand-evolvement.

\subsection{Used Clustering Algorithm}

Some clustering approaches need to know the number of clusters. Of course one could estimate a number of clusters
by considering the number of industries as well as the number of different locations for each industry, but this would
still be an approximation to the number, which by the way would get invalid by adding more companies.
Hierarchical algorithms have the advantage that they do not need to know the number of used clusters. But this neither
solves the problem of getting good cluster because one would still have to figure out which of the multiple generated
clustering shemas should be used. So it is necessary to have a measurement of a cluster schema to find out which one
works best.

Furthermore the used clustering algorithms has to be exclusive and intrinsic. It would not be on purpose to find
characteristics on predefined groups but rather to define groups of companies. An exclusive approach would provide
the information to which cluster a company belongs.

% We will use a hierarchical clustering to group strongly connected companies.
%
% So our approach has to be, to get an evaluation of all companies according to their closeness to each other. This
% would also create some kind of clusters, but we would still have the information of closeness between each company.
% An appropriate datastructure to store this information is a graph.

The aim to explore and furthermore predict the need evolvement could be achieved by grouping strongly connected companies.
Companies that belong to one cluster should ideally have the same demands. To match the main thesis its important to find
correlations between closeness of companies and their needs. Especially its important that a cluster evolves exactly one same
need. This requirement makes to possible to allow predictions on a cluster's demand evolvement.

Therefore the approach will be to calculate the proximity between each of the companies. This makes it possible to
look for existing correlations and form clusters. A agglomerative hierarchical algorithm will be used to perform this.

As the algorithm produces different possible clusters a way to determine the best clusters is necessary. One clustercombination
has to fullfill the following characteristics:
\begin{itemize}
  \item All the clusters have a strong increase of exactly one demand each
  \item A cluster contains only companies that do not have the maximum possible proximity
\end{itemize}


\subsection{Calculate Proximity}
The calculation is performed on a randomly sampled subset of our total set of companies to save time and keep the amount
of produced realations as small as possible. The subset contains 1192 documents. This would result in a maximum count of
realtionships of 709.836, where each company has a relationship to each other, but not itself. Thats the result of following
function where n equals the number of companies.

\begin{center}
  { \Large #relations \LARGE = $\frac{n*(n-1)}{2}$}
\end{center}

For the proximity we take all of the features and weight them according to their influence. Regarding to the conclusions in chapter
\ref{companyFeatures} we assume that location and industry have a high weight whereas the company size does not have that much impact
on a company's buying behaviour.


\subsection{Cluster scoring}

To be able to evaluate which feature wheight is the best and which cluster combination of the set that emerges from the hierarchical
clustering it is necessary to have some characteristics to compare.

The first and most important one is the function score(X) that calculates how good a cluster is according to the fact whether
it strongly develops only one demand.

\begin{figure}[ht]
\includegraphics[scale=0.6]{goodCluster.png}
\includegraphics[scale=0.6]{badCluster.png}
\centering
\caption{A good and a bad example of a cluster's demand development}
\label{fig:clusterDemandDevelopment}
\end{figure}

Figure \ref{fig:clusterDemandDevelopment} shows a good demand development on the left and a bad one on the rigth. The used Dataset
of demand-posts covers 5 different products: CRM, ECOM, HCM and LVM. The percentage value says how many of the companies within
the according cluster raised a demand of the corresponding product.
The left distribution allows to conclude that the remaining companies in the cluster may also be interested in a HCM product.

\begin{figure}[ht]
  {
    \Large Let $X \subseteq \mathbb{Q}$ \newline
    \Large score(X) \LARGE = $\frac{ maxVal( ( max2(X) - avg2(X) ), 1 ) }{ ( max(X) - avg(X) ) }$ \newline \newline
    \large max(X) =  maximum value of set X \newline
    \large avg(X) =  average value of set X \newline
    \large max2(X) = $ max( X \textbackslash \{ y | y = max(X)\} )$ \newline
    \large avg2(X) = $ avg( X \textbackslash \{ y | y = max(X)\} ) $ \newline
  }
  \centering
  \caption{The scoring function}
  \label{fig:scoringFunction}
\end{figure}

The function score(X) returns a value that describes the ratio between the difference of the highest value and the
average to the difference of the second highest value and the average without the highest value. The closer the value
is to 0 the better the cluster.

If the highest value strongly differs from all the others than it has a high difference to the average value.
If the highest value is by far the highest than the difference between the second highest value and the average without
the higest value will small. To prevent a wrong result the denominator has to be at least 1 because otherwise the
whole value could be 0 even if the highest value does not have a high difference to the average.

To clarify the formula we are going to calculate for the two examples in figure \ref{fig:clusterDemandDevelopment}
The Left distribution:
\begin{center}
  score([4,4,90,3,4]) =  {\Large $\frac{ maxVal( ( 4 - 3,75 ), 1 ) }{ ( 90 - 21 ) } = \frac{ 1 }{ 69 } =$} 0.0144 \\
\end{center}

The right distribution:
\begin{center}
  score([28,23,23,24,6]) =  {\Large $\frac{ maxVal( ( 24 - 19 ), 1 ) }{ ( 28 - 20,8 ) } = \frac{ 5 }{ 7,2 } =$} 0.6944 \\
\end{center}


The left distribution has as expected a better value than the right distribution. To evaluate a whole cluster combination
the rating for each cluster gets calculated. All the ratings will then be averaged according to the clusters size.
So a good rating within a small cluster will not have as much impact as a good rating in a bigger sized cluster.

Other measurments to value a cluster combination are the total number of companies within the clusters or the
highest average of a products demand. The more companies covered, the more efficient the demand predictions are.
Also the higher the average covering is for the products, the more actively are the companies spreading demands
within a cluster. An average covering takes only the highest coverage from each of the existing clusters and averages
them. According to our exmaple in figure \ref{fig:clusterDemandDevelopment} we would calculate the average of 90 and 28.

























% {\small
% \begin{tabular}{ccccccc}
%   Clusters & Avg Rating & Level & High Avg & Big Cluster & Weight Sz In Lo & Tree depth \\
%     5 & 0.7969 & 28 & 38\% & 1049 & 0,0,1 & 777 \\
%     6 & 0.9860 & 67 & 22\% & 217 \(561\) &  1,0,0 & 284 \\
%     6 & 0.9860 & 67 & 22\% & 217 \(561\) &  0,0,1 & 284 \\
%     6 & 0.7001 & 47 & 48\% & 12 \(34\) & 1,1,1 & 59 \\
%     8 & 0.7375 & 51 & 45\% & 10 \(42\) & 2,4,1 & 61 \\
%     8 & 0.8693 & 41 & 47\% & 9 \(39\) & 2,2,1 & 50 \\
%     5 & 0.7125 & 10 & 39\% & 1086 \(1097\) & 2,8,1 & 63 \\
% \end{tabular}
% }

	\clearpage

	\section{Prototype}
For this thesis a tool was developed that easily allows to cluster companies by different feature weights,
select the most appropriate cluster from all the cluster combinations emerged by a hierarchical clustering
and visualize this cluster and show the important information and statistics for each group.

\begin{figure}[ht]
\includegraphics[width=0.6\linewidth]{clusterVisualization.png}
\includegraphics[width=0.3\linewidth]{clusterStats.png}
\centering
\caption{A cluster visualization}
\label{fig:clusterVisualization}
\end{figure}

Figure \ref{fig:clusterVisualization} shows an example visualization of clusters produced by the tool.
Each circle represents an own cluster. The bigger each circle's radius, the more companies are contained within this cluster.
The distance between each cluster shows their distance towards each other. Circles with a red border include companies
that raised a demand in a timeframe that can be adjusted. The different coloring of the clusters illustrates the dominant industry.
A click on a cluster gives an overview of the most important
values, as what needs were raised, what is the clusters or the overall demand distribution for all products or what is
the accordingly cluster score.

	\clearpage

	\section{Evaluation}

\subsection{Correlation of company closeness and need development}

% why or why not may companies raise certain needs ~
\subsection{Treats for raising certain needs}

	% \clearpage

	\section{Conclusion}
This thesis presented an approach to group companies, evaluate different clusters regarding
to their need development and predict demands of companies.

This thesis could not give the final answer to the problem, it is further more a
proof of concept that the primary objective to identify and current and future product
needs is possible.

The opportunities are promising. Knowing companie's demands beforehand would create a big
advantage to the ones using it.

\section{Future Work}
As this thesis is more a proof of concept than a final solution the main aspect of the
future work will be to improve the clustering by refering to more metrics as already mentionen
in section \ref{section:improvingResult}. Therefore it could be useful to also mention additional
datasources like the Compustat database which provides financial, statistical and market information
on companies throughout the world.

Another important part would be to enable the tool to work on huge datasets and implement the prediction
part to gain value from the obtained conclusions.

	\clearpage

	%%% BIBLIOGRAPHY
	%\bibliographystyle{babunsrt3-fl}
	\addcontentsline{toc}{section}{Bibliography}
	\bibliographystyle{babunsrt-fl}
	\bibliography{projektbib}

\end{document}
